\chapter{Ensemble Learning}\label{sec:ensembles}
From our prior research \cite{kang2016ensemble,an2016map}, we know that ensembles frequently beat vanilla classifiers.
As a consequence, we decided to try out ensembles on our data.
From our set of 154 classifier data view runs, we  selected the 20 best runs (six of which
were presented in detail in Section \ref{sec:ml}). We constructed
a variety of ensembles of size 2 to 9 classifiers in each 
from these runs, and via attrition zeroed in on
the best performing ones. Our measure of performance of an ensemble
was straightforward:

\begin{quote}
\textit{the \textsf{MA-MAE} of the ensemble must
be higher than $0.851$, the \textsf{MA-MAE} of our best standalone method}
(\textsf{SVM-Init-Ada-CARS}).
\end{quote} 


Our classifier ensembles were constructed in a straightforward way. Each ensemble
consisted of a subset of classifiers from our list of 20
best runs. Each classifier in the ensemble
received an equal vote share (i.e., we did not attempt to weigh
classifiers differently). We devised six different voting schemes to determine
the ensemble prediction of the \textsf{Valence} based on the predictions
of the constituent classifiers.  These voting schemes are defined below.

\paragraph{Majority voting} A value of the \textsf{Valence} is selected
if it was predicted by the majority (at least half) of classifiers in the
ensemble. If such value does not exist, this method picks the \textit{most common} \textsf{Valence} in the training set\footnote{In our training set,
this was \textsf{Valence=MILD}.}.

\paragraph{Plurality voting} Given a parameter \textit{min\_votes}, a
specific value of the \textsf{Valence} is selected if it is the most
common value and more than \textit{min\_votes} classifiers in the
ensemble predict it. If such value does not exist, 
this method picks the \textit{most common} \textsf{Valence} in the training set.

\paragraph{Majority\_favor\_MODERATE voting} 
This scheme selects the majority
value of predicted \textsf{Valence} if one exists, 
the same way the \textit{majority} scheme works.
However, if a \textit{majority} value does not exist, this voting
scheme favors \textsf{Valence = MODERATE}: it selects this value if \textit{at least one}
classifier predicts it. If no classifier predicts \textsf{Valence = MODERATE}, 
this scheme defaults to the most common \textsf{Valence} in the training set.

\paragraph{Plurality\_favor\_MODERATE voting} 
This scheme selects
the plurality value of \textsf{Valence} if it is predicted by more than
\textit{min\_votes} votes. If such value does not exist, but at least
one classifier predicts \textsf{Valence = MODERATE}, this scheme selects this value.
If no \textsf{Valence = MODERATE} prediction exists in the ensemble, 
the voting scheme defaults to the most common \textsf{Valence} in the training set.

\paragraph{Simple Round voting}  This voting scheme simply finds
the average prediction value among the ensemble classifiers,
and rounds it to the nearest \textsf{Valence} value.

\paragraph{Tuned Round voting}

Our most complicated voting scheme acts like another hyper-parametrization search.

The general idea behind this method is fairly straightforward, although the implementation
does require some explanation.  The \textit{simple round voting} scheme assumes
that the average valence of $2.6$ (out of $3$) points to the class \textsf{Valence = SEVERE},
as $2.6$ is greater than the midpoint between the numeric \textsf{Valence} scores for
\textsf{MODERATE} and \textsf{SEVERE} classes. However, \textsf{Valence} (despite
how we choose to treat it on occasion) is \textbf{not} a continuous variable, but
an ordinal one.  Therefore, $0.5$, $1.5$ and $2.5$ \textit{do not have to be} the threshold
values separating the neighboring valence classes.  What should these values be?
Well, we can treat this as yet another hyperparameter tuning problem, and find
such values of the three threshold parameters that optimize the \textsf{MA-MAE} score.

For our experiments we used the following threshold sets, which give rise to
a search space of 343 possibilities.

\begin{itemize}
    \item \textsf{NONE/MILD threshold}: \textsf{0.25, 0.3, 0.4, 0.5, 0.6, 0.7, 0.75}
    \item \textsf{MILD/MODERATE threshold}: \textsf{1.25, 1.3, 1.4, 1.5, 1.6, 1.7, 1.75}
    \item \textsf{MODERATE/SEVERE threshold}: \textsf{2.25, 2.3, 2.4, 2.5, 2.6, 2.7, 2.75}
\end{itemize}

%% TODO: insert algorithm here?

\begin{algorithm}[h]
\SetKwData{Combinations}{Combinations}
\SetKwData{MAMAE}{MA\_MAE}
\SetKwData{Array}{Array}
\SetKwData{RowSum}{RowSum}

\SetKwInOut{truth}{truth}
\SetKwInOut{predictions}{predictions}
\SetKwInOut{k}{k}
\SetKwInOut{thresholds}{thresholds}
\SetKwInOut{Output}{output}

\truth{correct Valence scores for each document}
\predictions{predictions of all trained classifiers to attempt to construct an ensemble for}
\k{max size of ensemble}
\thresholds{boundary thresholds to use as triplets}
\Output{The best ensemble from training data}
\BlankLine

$best\_ensemble \gets null$ \;
$best\_score \gets 0$ \;
\For{$i$ in $2..k$}{
    \For{$candidate\_ensemble$ in \Combinations{$predictions$, $i$}}{
        $predictions \gets $ \RowSum{$candidate\_ensemble$} \;
        \For{$none, mild, moderate$ in $thresholds$}{
            $rounded \gets [$ \\
                $SEVERE$ if $v > moderate$ else \\
                $MODERATE$ if $v > mild$ else \\
                $MILD$ if $v > none$ else $NONE$ \\
                for $v$ in $predictions$ \\
            $]$ \;
            $score \gets$ \MAMAE{$rounded$, $truth$} \;   
            \If{$score > best\_score$}{
                $best\_score \gets score$ \;
                $best\_ensemble \gets ensemble, none, mild, moderate$ \;
            }   
        }
    }
}
\caption{Tuned Round Voting} \label{alg:tunedround}
\end{algorithm}

To run all our ensembles through this voting mechanism we would have to generate
in excess of 13.3 million combinations. To achieve this,  the \textit{tuned round voting}
ensemble algorithm was parallelized onto a \textsf{c4.4xlarge} Cloud Instance on Amazon AWS. In addition, the mean votes of each of the 38,760 candidate ensembles were pre-computed using OpenBLAS \cite{open-blas}. Nonetheless, the entire computation took 8 hours with classifier ensembles of no larger than 6, whereas \textit{all}  Majority and Plurality schemes up to 9 completed on a single core in less time. %% to be honest, i don't 100% remember how long it took but it was a while. Less than an entire 5 - 9 overnight. 

\section{Submitted Ensembles}

Among the multitude of voting ensembles, we selected the five top performers (all
providing us with 1.5 -- 3\% of lift over the best individual
classifier) shown in Table \ref{tab:ourEnsembles}. Notably,
all these ensembles used the \textit{tuned voting} ensemble voting, confirming to us that the tuning of the thresholds
separating the neighboring \textsf{Valence} classes was a
useful procedure. The \textsf{MA-MAE} scores
reported in Table \ref{tab:ourEnsembles} were computed
over the 325-record training set.

As we could only submit three guesses, we had to make our final
choices from these five ensembles. 
\textit{
We selected ensembles \textbf{A}, \textbf{B} and \textbf{C} for official submission.}

\textsf{Ensemble B} consisted only of the Random Forest regressor
run on the full data view, and our most accurate standalone
classifier, SVM-initialized Na\"{i}ve Bayes AdaBoost on the Class Association
Rules data view. It also was the best performer on the training set.

\textsf{Ensembles A} and \textsf{C} were selected to diversify our 
pool of guesses. \textsf{Ensemble A} was
selected as the most accurate ensemble \textit{that did
not feature classifiers trained on Class Association Rules alone}.
We chose \textsf{Ensemble C} over \textsf{Ensemble E} because
in a secondary run on the 108 \textsf{annotated\_by\_1} records
\textsf{Ensemble E} had a drop in accuracy that worried us.
Additionally, \textsf{Ensemble C} was far better than the other
ensembles in properly recognizing the \textsf{Valence = SEVERE} class.




% Ensemble A: 6 Classifiers. Tuned Voting Strategy with Thresholds at \textsf{0.7}, \textfs{1.6}, and \textfs{2.25}. 'RF-full', 'RF-reg-clf-full', 'RF-Reg-full', 'Lin-SVM-chi2-best', 'Lin-SVM-anova-best', 'DNN-full'.

% Ensemble B: 2 Classifiers. Tuned Voting Strategy with Thresholds at \textsf{0.7}, \textfs{1.7}, and \textsf{2.3}. 'RF-Reg-full', 'SVM-Init-Ada-CARS'

% Ensemble C: 6 Classifiers. Tuned Voting Strategy with Thresholds at \textfs{0.75}, \textsf{1.5}, and \textsf{2.25}. 'RF-Reg-full', 'Lin-SVM-chi2-best', 'RBF-SVR-mig-best', 'D&W-num', 'SVM-Init-rules', 'MNB-CARs'




\begin{table}
    \centering
    \begin{tabular}{|cllc|}
    \hline
    \textsf{Name}  & \textsf{Classifiers}& \textsf{Voting}&
    \textsf{MA-MAE}\\
    \hline
        & \textsf{RF-full}  &    & \\
        & \textsf{RF-reg-clf-full} &  &  \\
    \textsf{A}    & \textsf{RF-Reg-full} &   Tuned round & $0.865$\\
        & \textsf{Lin-SVM-chi2-best} &     (0.7,1.6,2.25)& \\
        & \textsf{Lin-SVM-anova-best} &  & \\
        & \textsf{DNN-full}& & \\
    \hline
    \textsf{B}    & \textsf{RF-reg-full} & Tuned round& $0.882$ \\
        & \textsf{SVM-Init-Ada-CARS} &   (0.7,1.7,2.3)& \\
    \hline
        & \textsf{RF-Reg-full} &  & \\
        & \textsf{Lin-SVM-chi2-best} & & \\
    \textsf{C}    & \textsf{RBF-SVR-mig-best} & Tuned round & $0.865$  \\
        & \textsf{D\&W-num} &   (0.75,1.5,2.25)& \\
        & \textsf{SVM-Init-Ada-CARS} & & \\
        & \textsf{MNB-CARs} & & \\
    \hline
         &  \textsf{RF-reg-full} & & \\
    \textsf{D}     &  \textsf{Lin-SVM-chi2-best} & Tuned round & $0.850$ \\
         &  \textsf{Lin-SVM-ANOVA}&  (0.5,1.3,2.3) & \\
         &  \textsf{DNN-full}  & & \\
    \hline
        & \textsf{RF-Reg-full} &  & \\
    \textsf{E}    & \textsf{SVM-Init-Ada-CARS} & Tuned round & $0.866$ \\
        & \textsf{DNN-full} &  (0.7,1.4,2.4) & \\
        & \textsf{Lin-SVM-chi2-best} & & \\
    \hline

    \end{tabular}
    \caption{Top five voting ensembles. The \textsf{MA-MAE} value
    is computed on the 325-record training set.}
    \label{tab:ourEnsembles}
\end{table}





\begin{table}
\centering
    \begin{tabular}{|l|c|c|c|c|}
   \hline
   \textsf{ \cellcolor{gray!15} Ensemble A } & \textsf{ Pred NONE } & \textsf{ Pred MILD } & \textsf{ Pred MODERATE } & \textsf{ Pred SEVERE } \\ 
    \hline
    \textsf{ NONE } & \cellcolor{gray!15} 34 & 10 & 1 & 0 \\ 
    \textsf{ MILD } & 9 & \cellcolor{gray!15} 102 & 17 & 2 \\ 
    \textsf{ MODERATE } & 1 & 15 & \cellcolor{gray!15} 48 & 18 \\ 
    \textsf{ SEVERE } & 0 & 3 & 19 & \cellcolor{gray!15} 46 \\ 
   \hline
   \textsf{ \cellcolor{gray!15} ENSEMBLE B } & \textsf{ Pred NONE } & \textsf{ Pred MILD } & \textsf{ Pred MODERATE } & \textsf{ Pred SEVERE } \\ 
    \hline
    \textsf{ NONE } & \cellcolor{gray!15} 29 & 15 & 1 & 0 \\ 
    \textsf{ MILD } & 8 & \cellcolor{gray!15} 116 & 4 & 2 \\ 
    \textsf{ MODERATE } & 1 & 20 & \cellcolor{gray!15} 54 & 7 \\ 
    \textsf{ SEVERE } & 0 & 1 & 20 & \cellcolor{gray!15} 47 \\ 
   \hline
   \textsf{ \cellcolor{gray!15} ENSEMBLE C } & \textsf{ Pred NONE } & \textsf{ Pred MILD } & \textsf{ Pred MODERATE } & \textsf{ Pred SEVERE } \\ 
    \hline
    \textsf{ NONE } & \cellcolor{gray!15} 31 & 14 & 0 & 0 \\ 
    \textsf{ MILD } & 10 & \cellcolor{gray!15} 107 & 10 & 3 \\ 
    \textsf{ MODERATE } & 2 & 21 & \cellcolor{gray!15} 42 & 17 \\ 
    \textsf{ SEVERE } & 0 & 3 & 10 & \cellcolor{gray!15} 55 \\ 
    \hline
    \end{tabular}    
    \caption{Confusion Matrices on the 325 document training set for the 3 submitted Ensembles}
    \label{tab:EnsembleTrainConfusionMatrix}
\end{table}

\begin{table}
\centering
    \begin{tabular}{|l|c|c|c|}
    \hline
   \textsf{\cellcolor{gray!15} ENSEMBLE A} & \textsf{Precision} & \textsf{Recall} & \textsf{MAE} \\ 
   \hline
    \textsf{NONE} & 0.773 & 0.756 & \cellcolor{gray!15} 0.911 \\ 
    \textsf{MILD} & \cellcolor{gray!15} 0.785 & \cellcolor{gray!15} 0.785 & 0.885 \\ 
    \textsf{MODERATE} & 0.565 & 0.585 & 0.787 \\ 
    \textsf{SEVERE} & 0.697 & 0.676 & 0.877 \\ 
   \hline
   \textsf{\cellcolor{gray!15} ENSEMBLE B} & \textsf{Precision} & \textsf{Recall} & \textsf{MAE} \\ 
   \hline
    \textsf{NONE} & 0.763 & 0.644 & 0.874 \\ 
    \textsf{MILD} & 0.763 & \cellcolor{gray!15} 0.892 & \cellcolor{gray!15} 0.939 \\ 
    \textsf{MODERATE} & 0.684 & 0.659 & 0.823 \\ 
    \textsf{SEVERE} & \cellcolor{gray!15} 0.839 & 0.691 & 0.892 \\ 
   \hline
   \textsf{\cellcolor{gray!15} ENSEMBLE C} & \textsf{Precision} & \textsf{Recall} & \textsf{MAE} \\ 
   \hline
    \textsf{NONE} & 0.721 & 0.689 & 0.896 \\ 
    \textsf{MILD} & \cellcolor{gray!15} 0.738 & \cellcolor{gray!15}  0.823 & 0.900 \\ 
    \textsf{MODERATE} & 0.677 & 0.512 & 0.744 \\ 
    \textsf{SEVERE} & 0.733 & 0.809 & \cellcolor{gray!15} 0.922 \\ 
    \hline
    \end{tabular}
    \caption{Multi-class Precision, Recall and MAE on the 325 document training set for the 3 submitted Ensembles. Per-class MAE is normalized with the assumption that all predictions are maximally incorrect for each class.}
    \label{tab:EnsembleTrainingMetrics}
\end{table}

\begin{table}
\centering
    \begin{tabular}{|c|c|c|c|}
         \hline
         \textsf{ Ensemble } & \textsf{ MA-MAE } & \textsf{ ROC-AUC } & \textsf{ $R^2$ } \\
         \hline
         \textsf{ \cellcolor{gray!15} A } & 0.865 & 0.795 & 0.622  \\
         \textsf{ \cellcolor{gray!15} B } & \cellcolor{gray!15} 0.882 & \cellcolor{gray!15} 0.823 & \cellcolor{gray!15} 0.694  \\ 
         \textsf{ \cellcolor{gray!15} C } & 0.865 & 0.801 & 0.629  \\ 
         \hline
    \end{tabular}
    \caption{Summary metrics on the 325 document training set for each of the 3 submitted Ensembles. All applicable metrics are macro-averaged when necessary. Higher is better.}
    \label{tab:EnsembleTrainingMAMetrics1}
\end{table}

\begin{table}
\centering
    \begin{tabular}{|c|c|c|c|c|}

         \hline
         \textsf{ Ensemble } & \textsf{ Precision } & \textsf{ Recall } & \textsf{ F1-Score } & \textsf{ Accuracy } \\ 
         \hline
         \textsf{ \cellcolor{gray!15} A } & 0.705 & 0.701 & 0.703 & 0.708 \\
         \textsf{ \cellcolor{gray!15} B } & \cellcolor{gray!15} 0.762 & \cellcolor{gray!15} 0.722 & \cellcolor{gray!15} 0.738 & \cellcolor{gray!15} 0.757 \\ 
         \textsf{ \cellcolor{gray!15} C } & 0.717 & 0.708 & 0.709 & 0.723 \\ 
         \hline
    \end{tabular}
    \caption{Summary metrics on the 325 document training set for each of the 3 submitted Ensembles. All applicable metrics are macro-averaged when necessary. Higher is better.}
    \label{tab:EnsembleTrainingMAMetrics2}
\end{table}

%% "Table ?? shows the precision"
Table \ref{tab:EnsembleTrainConfusionMatrix} shows
the confusion matrices of the three submitted ensembles
on the 325-record training set. Table \ref{tab:EnsembleTrainingMAMetrics1}
shows the precision, recall and \textsf{MAE} for each \textsf{Valence}
class for each ensemble.  Table \ref{tab:EnsembleTrainingMAMetrics1}
shows the \textsf{MA-MAE} as well as the ROC-AUC and the \textsf{$R^2$}
metrics. Table \ref{tab:EnsembleTrainingMAMetrics2} shows overall
precision, recall, f-measure and accuracy of the ensembles.

%%% TO SKYLAR: why are MA-MAE scores different in Table 19 and
%%%            in the data from which I built Table 16?
%%%         Is it because Table 16 is actually on 433 records???
%%%         If yes - we need to fix text.


%% other things to notice: Regressors often had lower individual MA-MAE
% scores compared to Classifiers, but Regressors often worked better as part 
% of the Tuned Voting Ensemble

%% Tuned Voting Ensembles without any classifiers that focused on only rules
% did much better than Majority and Plurality schemes (a lift of 0.04 - 0.05). 
% This gap got much smaller once we added association rules towards the end
% of the competition.

\section{Test Results}



\begin{table}
\centering
    \begin{tabular}{|l|c|c|c|c|}
       \hline
       \textsf{ \cellcolor{gray!15} ENSEMBLE A } & \textsf{ Pred NONE } & \textsf{ Pred MILD } & \textsf{ Pred MODERATE } & \textsf{ Pred SEVERE } \\ 
       \hline
        \textsf{ NONE } & \cellcolor{gray!15} 20 & 10 & 1 & 0 \\ 
        \textsf{ MILD } & 12 & \cellcolor{gray!15} 66 & 5 & 3 \\ 
        \textsf{ MODERATE } & 2 & 16 & \cellcolor{gray!15} 23 & 5 \\ 
        \textsf{ SEVERE } & 1 & 3 & 10 & \cellcolor{gray!15} 39 \\ 
       \hline
       \textsf{ \cellcolor{gray!15} ENSEMBLE B } & \textsf{ Pred NONE } & \textsf{ Pred MILD } & \textsf{ Pred MODERATE } & \textsf{ Pred SEVERE } \\ 
       \hline
        \textsf{ NONE } & \cellcolor{gray!15} 20 & 11 & 0 & 0 \\ 
        \textsf{ MILD } & 4 & \cellcolor{gray!15} 68 & 13 & 1 \\ 
        \textsf{ MODERATE } & 2 & 19 & \cellcolor{gray!15} 21 & 4 \\ 
        \textsf{ SEVERE } & 0 & 8 & 7 & \cellcolor{gray!15} 38 \\ 
       \hline
       \textsf{\cellcolor{gray!15} ENSEMBLE C } & \textsf{ Pred NONE } & \textsf{ Pred MILD } & \textsf{ Pred MODERATE } & \textsf{ Pred SEVERE } \\ 
       \hline
        \textsf{ NONE } & \cellcolor{gray!15} 21 & 10 & 0 & 0 \\ 
        \textsf{ MILD } & 7 & \cellcolor{gray!15} 64 & 13 & 2 \\ 
        \textsf{ MODERATE } & 2 & 12 & \cellcolor{gray!15} 29 & 3 \\ 
        \textsf{ SEVERE } & 0 & 3 & 9 & \cellcolor{gray!15} 41 \\ 
       \hline
    \end{tabular}
    \caption{Confusion Matrices on the 216 document hidden test set for the 3 submitted Ensembles}
    \label{tab:EnsembleTestConfusionMatrix}
\end{table}

\begin{table}
\centering
    \begin{tabular}{|l|c|c|c|}
        \hline
       \textsf{\cellcolor{gray!15} ENSEMBLE A} & \textsf{Precision} & \textsf{Recall} & \textsf{MAE} \\ 
       \hline
        \textsf{NONE} & 0.571 & 0.645 & \cellcolor{gray!15} 0.871 \\ 
        \textsf{MILD} & 0.695 & \cellcolor{gray!15} 0.767 & 0.867 \\ 
        \textsf{MODERATE} & 0.590 & 0.500 & 0.739 \\ 
        \textsf{SEVERE} & \cellcolor{gray!15} 0.830 & 0.736 & 0.881 \\ 
       \hline
       \textsf{\cellcolor{gray!15} ENSEMBLE B} & \textsf{Precision} & \textsf{Recall} & \textsf{MAE} \\ 
       \hline
        \textsf{NONE} & 0.769 & 0.645 & 0.882 \\ 
        \textsf{MILD} & 0.642 & \cellcolor{gray!15} 0.791 & \cellcolor{gray!15} 0.890 \\ 
        \textsf{MODERATE} & 0.512 & 0.457 & 0.707 \\ 
        \textsf{SEVERE} & \cellcolor{gray!15} 0.884 & 0.717 & 0.855 \\ 
       \hline
       \textsf{\cellcolor{gray!15} ENSEMBLE C} & \textsf{Precision} & \textsf{Recall} & \textsf{MAE} \\ 
       \hline
        \textsf{NONE} & 0.700 & 0.677 & 0.892 \\ 
        \textsf{MILD} & 0.719 & 0.744 & 0.861 \\ 
        \textsf{MODERATE} & 0.569 & 0.630 & 0.794 \\ 
        \textsf{SEVERE} & \cellcolor{gray!15} 0.891 & \cellcolor{gray!15} 0.774 & \cellcolor{gray!15} 0.906 \\ 
       \hline
    \end{tabular}
    \caption{Multi-class Precision, Recall and MAE on the 216 document hidden test set for the 3 submitted Ensembles. Per-class MAE is normalized with the assumption that all predictions are maximally incorrect for each class.}
    \label{tab:EnsembleTestMetrics}
\end{table}

%%% TO ALEX: do we want a "delta" where we look at all our stats and compare the difference between our train and our test?

%%% TO SKYLAR: would be nice. I've done some spotcheck comparisons in
%%%            the text. E.g., Ensemble 3 kills SEVERE in both precision
%%%            and recall on the test set, while being so-so on 
%%%            training set.
%%% This is something we might add during revision.


\begin{table}
\centering
    \begin{tabular}{|c|c|c|c|}
        \hline
        \textsf{ Ensemble } & \textsf{ MA-MAE } & \textsf{ ROC-AUC } & \textsf{ $R^2$ } \\
        \hline
       \cellcolor{gray!15} \textsf{ A } & 0.837 & 0.776 & 0.534 \\ 
       \cellcolor{gray!15} \textsf{ B } & 0.833 & 0.763 & 0.539 \\ 
       \cellcolor{gray!15} \textsf{ C } & \cellcolor{gray!15} 0.863 & \cellcolor{gray!15} 0.799 & \cellcolor{gray!15} 0.629 \\ 
       \hline
    \end{tabular}
    \caption{Summary metrics on the 216 document hidden test set for each of the three submitted Ensembles. All applicable metrics are macro-averaged when necessary. Higher is better.}
    \label{tab:EnsembleTestSummaryMetrics1}
\end{table}

\begin{table}
\centering
    \begin{tabular}{|c|c|c|c|c|}
        \hline
        \textsf{ Ensemble } & \textsf{ Precision } & \textsf{ Recall } & \textsf{ F1-Score } & \textsf{ Accuracy } \\
        \hline
       \cellcolor{gray!15} \textsf{ A } & 0.671 & 0.662 & 0.664 & 0.685 \\ 
       \cellcolor{gray!15} \textsf{ B } & 0.702 & 0.652 & 0.671 & 0.681 \\ 
       \cellcolor{gray!15} \textsf{ C } & \cellcolor{gray!15} 0.720 & \cellcolor{gray!15} 0.706 & \cellcolor{gray!15} 0.712 & \cellcolor{gray!15} 0.718 \\ 
        \hline
    \end{tabular}
    \caption{Summary metrics on the 216 document hidden test set for each of the three submitted Ensembles. All applicable metrics are macro-averaged when necessary. Higher is better.}
    \label{tab:EnsembleTestSummaryMetrics2}
\end{table}



The results of running our three submitted ensembles on the
216-record test set are shown in Tables \ref{tab:EnsembleTestConfusionMatrix}---\ref{tab:EnsembleTestSummaryMetrics2}.
Table \ref{tab:EnsembleTestConfusionMatrix} shows the confusion
matrices of the three ensembles on the test set.

It is worth immediately noting, by comparing confusion
matrices in  Table \ref{tab:EnsembleTestConfusionMatrix}
to those in Table \ref{tab:EnsembleTrainConfusionMatrix}
(for the training set), that all three ensembles overall
performed as expected and did not overfit the training
set by much.  We note that all three ensembles
 excelled at \textit{not making huge mistakes}:
they had 10, 11, and 7 data points (respectively) with classification
error of $2$ or more. This compares with 7, 5, and 8 such
data points misclassified on (somewhat larger) training set.

All three ensembles exhibited very similar performance
on the 31 records with \textsf{Valence = NONE}.
\textsf{Ensemble A} tended to underrate \textsf{MILD}
cases, preferring to predict \textsf{Valence = NONE}
when it made a mistake.  \textsf{Ensembles B} and \textsf{C}
went in the other direction, overrating the majority
of mistakes on cases with \textsf{Valence=MILD}.

It is on cases with \textsf{Valence=MODERATE} and \textsf{Valence=SEVERE}
\textsf{Ensemble C} showed a clearly better performance, both
in terms of recall (correctly classifying 29 out of 46 \textsf{MODERATE}
cases and 41 out of 53 \textsf{SEVERE} cases), and in terms
of precision (keeping it above 50\% for \textsf{Valence=MODERATE},
and allowing for only 5 false positives for \textsf{Valence=SEVERE}).
These numbers, especially the precision for the
\textsf{Valence=SEVERE} class wound up actually
being better than the training set results (where
\textsf{Ensemble C} had 20 false positives and 55 true positives
in this)!

Table \ref{tab:EnsembleTestMetrics} shows precision, recall
and \textsf{MAE} for each valence class for each ensemble.
Table \ref{tab:EnsembleTestSummaryMetrics1} shows the overall
\textsf{MA-MAE}, \textsf{ROC-AUC} and \textsf{$R^2$} metrics for each
ensemble, while Table \ref{tab:EnsembleTestSummaryMetrics2} summarizes
precision, recall, f1-score and accuracy.

As seen from Table \ref{tab:EnsembleTestMetrics}, \textsf{Ensemble C}
wound up being the top scorer among our submissions.  
As we learned shortly after the submission period closed, 
\textsf{Ensemble C} \textit{wound up being the best overall predictor of
patient condition severity} in the entire CEGS N-GRID 2016 Shared Task in Clinical Natural Language Processing (Track 2).
