\chapter{Feature Selection}\label{sec:selection}

Because we now had thousands of features to consider, we developed a
feature selection procedure that subjected each feature in our dataset to a battery of tests. \textit{Features that failed
every single  test were eliminated from consideration}.  
The battery of tests is described in the following sections, followed by a brief analysis of the surviving features.

\section{Association Rule test} This decision procedure is a simple existence check: \textit{keep a feature if it
appears in the antecedent of at least one of the 628 Class Association Rules
in our dataset.}

\section{Statistical Tests}

We used 3 statistical tests to filter features. Each of our three methods works best with different feature types and captures different associations with a class labels.

\subsection{$\chi^2$ test for categorical features}
We ran a $\chi^2$ test \cite{chisquare} for each categorical feature against the
\textsf{Valence} variable.  This test checks whether there are sufficient
grounds to believe that a specific categorical feature is associated
with another categorical feature purely by coincidence.  

Suppose a feature is completely uncorrelated with a valence. This means
that the distribution of its categories will be very similar to that of 
the distribution of the ground truth labels. Now, suppose a \textbf{YES} value for a particular categorical feature always corresponds to a \textbf{MILD} valence. If we have sufficient number of \textbf{YES} observations, we can compute the likelihood that this happened purely by chance -- which would quickly become very low. We
 rejected any 
categorical feature whose $\chi^2$ test yielded a p-value higher than $0.05$.
The $\chi^2$ test was implemented by using \textsf{scipy}'s
\textsf{chisquare} function to compute the p-value of each categorical feature \cite{scipy}.


\subsection{ANOVA F-test for continuous features} 
ANOVA F-tests are used to test the significance of a regression model \cite{anova}.
While we used the $\chi^2$ test to test for potential significance of our 
categorical features, we used the multi-way ANOVA F-test for all numeric features.
For each feature tested, we separated the data into four subsets, based on the 
value of the target {Valence} attribute. We then randomly sampled from these four groups. We then tested the means and standard
deviations in each of the four subsets to see if they represented similar or
different distributions, and compared them across our multi-way samples to see if there is a statistical bias. Similarly to the $\chi^2$ test, we set rejected any numeric attribute whose ANOVA F-test
produced a p-value of more than $0.05$. We used
scikit-learn's \textsf{f\_classif} function to 
compute the multi-way ANOVA tests \cite{scikit-learn}. 

\subsection{Mutual Information Gain test (MIG)}

Mutual Information Gain is typically used in measuring the robustness of clustering methods.
In unsupervised problems, MIG  is measured by calculating \textbf{P(X, Y)} - the probability that two variables \textbf{X} and \textbf{Y} occur in the same cluster - compared to the probability \textbf{P(X) * P(Y)} of their occurring in the same cluster by random chance. If there is a clear dependence  %% or anti-dependence
between the two variables, then the probability of \textbf{P(X, Y)} will be higher %% or lower
than \textbf{P(X) * P(Y)}. Recent research shows that MIG provides an additional level of feature selection in the context of textual classification and clustering \cite{Xu07}. In the case of supervised feature selection, we compare the entropies and distributions of \textsf{Valence} vs. each feature
using $K$ Nearest Neighbors. At the time of the N-GRID Challenge, \textsf{scikit-learn} \cite{scikit-learn}
did not have a completed implementation of
\textsf{mutual\_info\_classif}, but it was in the process of being developed. We ported \textsf{scikit-learn}'s partial implementation into our system.

\section{Linear SVM Recursive Feature Elimination}
Our final test involved running \textsf{scikit-learn}'s version
of the Support Vector Machine (SVM) classifier with
a linear kernel \cite{cortes95} and observe whether the feature survived the
\textit{Recursive Feature Elimination} process implemented
within it. An advantage of using a linear SVM to find support vectors is that 
it provided our system with multivariate feature selection.
In addition, $\chi^2$ test and our Class Association Rules only worked on categorical features, 
while Mutual Information Gain requires a heuristic \cite{Xu07} to operate on continuous features.
The Linear SVM recursive feature elimination allowed us an additional test on
the continuous features in our dataset.

\section{Surviving Features}

Table \ref{tab:data3} contains the overview of the features that
survived this process: i.e., that passed successfully at least one of
the tests from the list above. We make a few observations here
about the final shape of the dataset. Only \textsf{LIWC} did
not contribute any features. All other means of enhancing
non-textual features provided meaningful contributions, 
with \textsf{cTakes}, original dataset, and, interestingly
enough, our cumulative scores accounting for the
majority of non-textual features.  All five \textsf{Common Value}
features also made it. Our manual work on documenting
medications resulted in 10 out of 47 medication
features kept. 

An unexpected outcome of 
this process was an essential depletion of directly word related
features from the dataset. Only 34 out of 300 Word2Vec dimensions were kept. For non-Word2Vec features, only a single unigram survived -- ``\textit{other}''. 
This implies that the categorical and yes/no responses have far more predictive power  than long form text.

\begin{table}
    \centering
    \begin{tabular}{|l|c|c|c|}
    \hline
    \textsf{Feature Category} & \textsf{\# Features} & \textsf{Feature Category} & \textsf{\# Features} \\
    \hline
    \textsf{TOTAL} & 788 & & \\
    \hline
    \textsf{Original} & 30 & \textsf{cTakes} & 40 \\
    \textsf{Cumulative scores} & 34 & \textsf{Common Value} & 5 \\
    \textsf{Medications} & 10 & \textsf{Word2Vec} & 34 \\
    \textsf{SentEmotion} & 6  & \textsf{CAR} & 628 \\
    \textsf{Unigrams} & 1 & & \\
     \hline
    \end{tabular}
    \caption{Description of the final set of features remaining in our
    operational dataset after the feature selection (pruning) step.}
    \label{tab:data3}
\end{table}


