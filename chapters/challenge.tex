\chapter{The Challenge}\label{sec:challenge}

The specification of Track 2 of the CEGS N-GRID 2016 Shared Task in Clinical Natural Language Processing  (also called the RDoC for Psychiatry Challenge) presented 
the goal of this particular track of the challenge as:
\begin{quote}
\textsl{``Determine symptom severity in a domain for a patient, based on information included in their initial psychiatric evaluation. The domain has been rated on an ordinal scale of 0-3. There is one judgment per document, and one document per patient."}\cite{N-GRID}
\end{quote}

Research Domain Criteria (RDoC) is a framework for facilitating the study of human behavior,
both normal and abnormal in various clinical domains. The RDoC provided the data,
originally collected by Partners Healthcare Inc. and the Neuropsychiatric Genome-Scale and RDoC
Individualized Domains (N-GRID) project at the Harvard Medical School \cite{N-GRID}.  The data
was released to the challenge participants under a strict set of Rules of Conduct and the Data Use Agreement.

As shown in Table \ref{tab:data1}, a total of 649 records were released,  broken into a training set of 433 files and 
a test set of 216 files with no ground truth --- the latter released two days before
the submissions were due. The initial release of the 433 patient records was broken into two categories: a suggested training set of 325 files, and 108 files
called \textsf{annotated\_by\_1}. Since the contest organizers discouraged us from using the records from the \textsf{annotated\_by\_1} set as training set data, we focused 
most of our efforts on the 325-record training set.

Each record, originally stored in a single XML file, represented the information from the
initial psychiatric consultation of a single patient. For each record in the training set the challenge organizers supplied the ground truth
about the severity of the patient's psychiatric condition. This information was stored in
a single feature of the data files called \textsf{Valence}.  Table \ref{tab:valence} shows the
scale on which the patients' conditions were evaluated.  The judgment contained in the \textsf{Valence}
field came from a clinical expert and was based solely on the symptoms and medical, social, 
mental health, and family history captured in the provided data.

\begin{table}
    \centering
    \begin{tabular}{|l|l|}
    \hline
    Total number of records released & 649  \\
    \hline
    Number of records in suggested training set    & 325  \\
    \hline
    Number of records in additional training set    & 108  \\
    \hline
    Number of records in test set   & 216 \\
    \hline
    \end{tabular}
    \caption{Overview of released data.}
    \label{tab:data1}
\end{table}


\begin{table}
\centering
    \begin{tabular}{|l|l|}
    \hline
    \textsf{Value} & \textsf{Meaning}\\
    \hline
    $0$  & \textsf{NONE}\\
    $1$  & \textsf{MILD}\\
    $2$  & \textsf{MODERATE}\\
    $3$  & \textsf{SEVERE}\\
    \hline
    \end{tabular}
    \caption{The scale of the target \textsf{Valence} variable in the N-GRID challenge training set data.}
    \label{tab:valence}
\end{table}

The  XML files provided by the challenge organizers contained both structured data, which documented
demographic information, mental health history, education, employment, financial status, family history of mental health, medical history, prescription and recreational drug use, and a few other categories of information; along with unstructured, free-form textual data, which documented self-reported symptoms and attending psychiatrists' notes on the patient and their condition. The data was in its raw, originally recorded form, and contained numerous typos, missing attributes, inconsistent use of abbreviations, and freeform text. The data was also anonymized.

Figure \ref{fig:sampleRecord} shows a \textit{notional (not real) record} created by us to illustrate the nature
of the data --- it contains no
information from the N-GRID dataset. However, this notional record
can give the reader an idea of the type of information that the teams had access to while working on the challenge
(as an aside, we note that the actual records contained significantly more data in them).

\begin{figure}[t]
\begin{tabular}{|l|}
\hline
\texttt{RAW DATA}\\
\texttt{    Name: John Doe}\\
  \texttt{  Age: 42}\\
\texttt{    Sex: Male}\\
\\
    \texttt{Referred by Emergency Services}
\\    
\texttt{    Has difficulty remembering if he has taken prescription drugs.}\\ \texttt{Accidental overdose.}
\\    
\texttt{    Referral Notes: Patient exhibits short-term memory loss}\\
\texttt{Mixed alcohol with
    prescription. Stayed overnight.}\\
\texttt{Found bruises on shoulders - possibly from falling.}\\
\\   
  \texttt{  DEPRESSION: YES}
  \\
  \texttt{OCD: No}
\\
\texttt{    PANIC: Yes}
\\    
\texttt{    Prescriptions:}\\
    \texttt{ Advil (3 times a day)}\\
\\         
\texttt{    Formulation:}
\\
\texttt{    Patient has history of anxiety and bipolar.}\\
\\    
\texttt{    Recommendations:}
\\
\texttt{    Change medication to Alpazolam.}
\\
\texttt{    Require additional visit in 2 weeks.}
\\    
\hline
\end{tabular}
\caption{A synthetic record illustrating the type of clinical medical records data contained in the
dataset released for Track 2 of the N-GRID challenge.}
\label{fig:sampleRecord}
\end{figure}

Table \ref{tab:data2} contains a rough breakdown of the types of features found in the original data. 
Table
\ref{tab:textdata} contains the list of features from the original data that 
were deemed by our team to be free-form text.


%%% Moving the TEXT fields into a separate table.

\begin{table}
{\small
    \centering
    \begin{tabular}{|l|c|c|c|}
    \hline
    \textsf{Feature Type} & \textsf{\# features} & \textsf{Feature Type} & \textsf{\# of features}  \\
    \hline
    \textsf{All features} &       102    & &   \\ 
    \hline
    \textsf{Demographic information} & 3 & \textsf{Family History} & 4 \\
    \textsf{Harming Others Or Self} & 4 & \textsf{Symptom Denial} & 8 \\
    \textsf{Mental Health Symptoms} & 18 & \textsf{Owns Firearms} & 1 \\ 
    \textsf{Drug, Caffeine and Alcohol Use} & 5 & \textsf{Legal History} & 1 \\
    \textsf{Independence of Daily Activities} & 9 & \textsf{Appearance} & 14 \\
    \textsf{Marital Status and Abuse} & 3 & \textsf{Military History} & 2 \\
    \textsf{Employment and Finances} & 4 & \textsf{Mental Health} & 18 \\
    \textsf{If Underage, Legal Guardian} & 2 & \textsf{Physical Health} & 6 \\
    \hline

    \end{tabular}
    \caption{Breakdown of features in the original N-GRID 2016 challenge Track 2 dataset by category.}
    \label{tab:data2}
}
\end{table}
  
\begin{table}
    {\small
    \centering
    \begin{tabular}{|c|c|}  
    \hline
    \multicolumn{2}{|c|}{\textsf{Free-form Entries}} \\
    \hline
    \textsf{Childhood History} & \textsf{History of Present Illness and Precipitating Events}  \\
    \textsf{Previous Treatments} & \textsf{Prior Medication Side-effects} \\
    \textsf{Current Medication} & \textsf{Chief Complaint (Patient's Own Words)}  \\ %%% like how and where they grew up %%%
    \textsf{Interpersonal Concerns} & \textsf{Education}  \\
    \textsf{Family Living Situation} & \textsf{Protective Factors}  \\
    \textsf{Risk Factors} & \textsf{Actions Taken}  \\
    \textsf{Formulation} & \textsf{Level of Care} \\
    \hline
    \end{tabular}
    \caption{List of free-form text fields found in the original data for Track 2 of the N-GRID 2016 challenge.}
    \label{tab:textdata}
    }
\end{table}


The results of the challenge were evaluated using the a variant of the \textsf{Mean Absolute Error (MAE)} metric.  Given a vector $\mathbf{v} = (v_1,\ldots,v_n)$ of ground truth values and a prediction
vector $\mathbf{p} = (p_1,\ldots, p_n)$, the \textsf{MAE} of the prediction is computed as 
$$ MAE(\mathbf{p},\mathbf{v}) = \frac{1}{n}\sum_{i=1}^{n}| p_i - v_i|$$

The \textsf{MAE} was altered by the challenge organizers so that the
score was in the range of [0, 1], where 1 indicated a perfect score. The new \textsf{MA-MAE} (\textsf{Macro-Averaged Mean Absolute Error}) measure was computed
by splitting the set of records into four categories (one per ground truth \textsf{Valence} value), computing the \textsf{MAE} for each of the four subsets independently, and combining
the computed \textsf{MAE}s into a weighted sum. The normalizing factors
for each of the component \textsf{MAE} values are the highest possible errors that
can be achieved for a data point with the given \textsf{Valence} value
($3$ for \textsf{Valence=0} and \textsf{Valence=3}, and $2$ for
\textsf{Valence =1} and \textsf{Valence = 2}).
To make the computed value correspond to the
\textit{higher is better} intuition, the computed weighted sum was subtracted from $1$.
The formula for computing the N-GRID Challenge version of \textsf{MAE} is:

\begin{figure}[H]
$$ MA\_MAE(\mathbf{p},\mathbf{v}) = 1 - 
    \frac{(\frac{MAE(\mathbf{P0}, \mathbf{V0})}{3} + 
     \frac{MAE(\mathbf{P1}, \mathbf{V1})}{2} + 
     \frac{MAE(\mathbf{P2}, \mathbf{V2})}{2} + 
     \frac{MAE(\mathbf{P3}, \mathbf{V3})}{3})}{4} $$
 \caption{Formula for computing Macro-Averaged Mean Average Error (MA-MAE)}
 \end{figure}


Every team participating in the challenge was allowed to submit up to three final guesses. Each guess was an XML file containing a single value which was the predicted valence for the specified clinical record from the provided \textit{test set}.
